\documentclass[DIV=13,10pt]{scrartcl}
\usepackage[utf8]{inputenc}
\usepackage[margin=2cm, footskip=.8cm]{geometry} % reduce page margin

\usepackage{pdflscape} % to switch to landscape mode
% \usepackage[ngerman]{babel}
% \usepackage[ngerman]{babel}
%%%%% Images %%%%%
\usepackage{graphicx}
% \usepackage{svg} % allow using .svg images
\usepackage{float} % allows abritrary figure positions using option [H]
\usepackage{wrapfig} % allows text wrapping around figures
\usepackage{subfig} % images next to each other
\usepackage{caption}                 % sans serif image captions #part 1
\captionsetup{font=sf, labelfont=bf} % sans serif image captions #part 2

\usepackage{url}
\usepackage[table]{xcolor}
% \usepackage{subfigure}
\usepackage{enumerate}

\usepackage{centernot} % um Dinge zentriert durchzustreichen
\usepackage{paralist}

\usepackage{amssymb}
    \let\oldemptyset\emptyset
    \let\emptyset\varnothing
\usepackage{mathtools} %includes amsmath
\usepackage{amsthm}
\usepackage{mleftright}
\usepackage{mathrsfs} %für Befehl mathscr
\usepackage{commath}
\usepackage{relsize} %für mathsmaller
\usepackage{xfrac}
\usepackage{nicefrac}
\usepackage{mathdots} %für iddots
\usepackage{csquotes} % for quoting and bibliography references

%%% generate hyperlinks in the document
\usepackage{hyperref}
\hypersetup{
    colorlinks = true,
    urlcolor = blue, % color of online references
    linkcolor = mydarkblue} % color of local references

\usepackage[backend=bibtex, style=alphabetic]{biblatex}   % bibliography
\addbibresource{sources.bib}
\renewcommand*{\bibfont}{\sffamily} % change bibliograpgy font 


\allowdisplaybreaks % allow page breaks in align
\setlength{\parindent}{0mm} % remove indentation at the start of new paragraphs

%%%%% Code Listings %%%%%
% provides commands for including code (python, latex, ...)
\usepackage{listings}
\definecolor{keywords}{RGB}{255,127,33}  %orange
\definecolor{comments}{RGB}{150,150,150} %medium grey
\definecolor{blue}{RGB}{0,0,255}
\definecolor{fav_blue}{RGB}{85,136,255}      %blue
\definecolor{green}{RGB}{0,168,0}        %green
\definecolor{grey}{RGB}{50,50,50}        %dark grey
\definecolor{light_grey}{RGB}{245,245,245}%light grey
\lstset{language=Python, 
        basicstyle = \ttfamily\small, 
        keywordstyle = \color{keywords},
        commentstyle = \color{comments},
        stringstyle = \color{green},
        showstringspaces = false,
        identifierstyle = \color{grey},
        backgroundcolor = \color{light_grey},
        xleftmargin = 10pt,  %text margin left
        xrightmargin = 10pt, %text margin right
        framexleftmargin = 5pt,  %frame margin left
        framexrightmargin = 5pt, %frame margin right
        numbers = left
        }

%%% the following produces a new environment like array but with automatic page breaks
\usepackage{array,longtable}
\usepackage{arydshln} % allow dashed table lines
\setlength{\dashlinegap}{2pt}
\setlength{\dashlinedash}{2pt}
\newcolumntype{C}{>{\(\displaystyle}c<{\)}}  % automatic math mode, centered
\newcolumntype{R}{>{\(\displaystyle}r<{\)}}
\newcolumntype{L}{>{\(\displaystyle}l<{\)}}
\newcolumntype{q}{>{\(}c<{\)}}  % automatic math mode, centered
\newcolumntype{s}{>{\(}r<{\)}}
\newcolumntype{e}{>{\(}l<{\)}}
\setlength\tabcolsep{5pt}     % match value of \arraycolsep
\renewcommand{\arraystretch}{1.2} % increased distance between lines in table and align environments

\usepackage{chngcntr} % make table numbers depend on section
\counterwithin{table}{section}
\usepackage{multicol}

%%% packages for pseudocode:
\usepackage{algorithm}
\usepackage{algorithmicx}
\usepackage{algpseudocode}

%%% TODO explain packages
% \usepackage{fancyhdr}
% \usepackage{lastpage}
% \pagestyle{fancy}
% \fancyhf{}
% \chead{Sebastian Jost}

%%%%% Math commands %%%%%

\DeclarePairedDelimiterX{\nor}[1]{\lVert}{\rVert}{#1}
\newcommand{\C}{\mathbb{C}}
\newcommand{\R}{\mathbb{R}}
\newcommand{\Q}{\mathbb{Q}}
\newcommand{\Z}{\mathbb{Z}}
\newcommand{\N}{\mathbb{N}}
\newcommand{\E}{\mathbb{E}}
\newcommand{\Pb}{\mathbb{P}}
\usepackage{bbm} % für mathbbm
\newcommand{\indf}[1]{\mathbbm{1}_{#1}} % Indikator Funktion

\newcommand{\bci}[1][k]{\bigcup_{#1=1}^{\infty}}
\newcommand{\too}[0]{\longrightarrow}
\newcommand{\limn}[0]{\lim_{n\to \infty}}

\newcommand{\eps}[0]{\varepsilon}
\newcommand{\cnot}[0]{\centernot} % \cnot as short version of \centernot
\newcommand{\ot}[0]{\leftarrow}
\newcommand\numberthis{\addtocounter{equation}{1}\tag{\theequation}} % add numbering manually

%%% Verteilungen
\DeclareMathOperator{\Exp}{Exp}
\DeclareMathOperator{\Geo}{Geo}
\DeclareMathOperator{\Bin}{Bin}
\DeclareMathOperator{\id}{id}

%%% Varianz und Kovarianz
\DeclareMathOperator{\Var}{Var}
\DeclareMathOperator{\Cov}{Cov}

\DeclareMathOperator{\diag}{diag}

\include{colors}

% \title{Vergleich von Architekturen und Parametern neuronaler Netzwerke}
\title{Result tables for experiments with cAdam}
\author{Sebastian Jost}
\date{\today}

\let\oldsubsubsection=\subsubsection
\renewcommand{\subsubsection}{%
  \filbreak
  \oldsubsubsection
}
% use bibliography file
\bibliography{sources}
% \bibliographystyle{abbrv}

\begin{document}
\sffamily
% \input{title_page}
\tableofcontents
% \newpage
\begin{landscape}



%%%%%%%%%%%%%%%%%%%%%%%%%%%%%%%%%%%%%%%%%%%%%%%%%%%%%%%%%%%%%%%%%%%%%%%
\section{new experiments: chemReg with cAdam}
\subsection{new experiments:chemReg with cAdam sorted by val MAE}
\input{new_experiments/final_val_loss_mae_avg_chemreg_cadam_best.tex}

\input{new_experiments/final_val_loss_mae_avg_chemreg_cadam_ratios.tex}

\input{new_experiments/final_val_loss_mae_avg_chemreg_cadam_worst.tex}

%%%%%%%%%%%%%%%%%%%%%%%%%%%%%%%%%%%%%%%%%%%%%%%%%%%%%%%%%%%%%%%%%%%%%%%
\section{MNIST Bachelorthesis new result tables}
\subsection{accuracy}
\input{new_experiments/final_val_accuracy_avg_mnist_bsc_best.tex}

\input{new_experiments/final_val_accuracy_avg_mnist_bsc_worst.tex}

\input{new_experiments/final_val_accuracy_avg_mnist_bsc_ratios.tex}

%%%%%%%%%%%%%%%%%%%%%%%%%%%%%%%%%%%%%%%%%%%%%%%%%%%%%%%%%%%%%%%%%%%%%%%
\section{MNIST Adam variant comparison}
\subsection{revised, longer experiments, 1 run}
\subsection{training time}
\input{Adam_variants_revised/training_time_avg_cadam_variants_best.tex}

\input{Adam_variants_revised/training_time_avg_cadam_variants_worst.tex}

\input{Adam_variants_revised/training_time_avg_cadam_variants_ratios.tex}
\subsection{accuracy}
\input{Adam_variants_revised/test_accuracy_avg_cadam_variants_best.tex}

\input{Adam_variants_revised/test_accuracy_avg_cadam_variants_worst.tex}

\input{Adam_variants_revised/test_accuracy_avg_cadam_variants_ratios.tex}


%%%%%%%%%%%%%%%%%%%%%%%%%%%%%%%%%%%%%%%%%%%%%%%%%%%%%%%%%%%%%%%%%%%%%%%
\section{MNIST with \% differences}
\subsection{training time}
\input{MNIST_revised/training_time_avg_mnist_revised_best.tex}

\input{MNIST_revised/training_time_avg_mnist_revised_worst.tex}

\input{MNIST_revised/training_time_avg_mnist_revised_ratios.tex}

\subsection{accuracy}
\input{MNIST_revised/final_val_accuracy_avg_mnist_revised_best.tex}

\input{MNIST_revised/final_val_accuracy_avg_mnist_revised_worst.tex}

\input{MNIST_revised/final_val_accuracy_avg_mnist_revised_ratios.tex}

%%%%%%%%%%%%%%%%%%%%%%%%%%%%%%%%%%%%%%%%%%%%%%%%%%%%%%%%%%%%%%%%%%%%%%%
\section{ChemRegB with \% differences}
\subsection{training time}
\input{chemRegB_14042023/training_time_avg_chemreg_adam_best.tex}
\input{chemRegB_14042023/training_time_avg_chemreg_adam_worst.tex}
\input{chemRegB_14042023/training_time_avg_chemreg_adam_ratios.tex}
\subsection{validation loss}
\input{chemRegB_14042023/final_val_loss_mae_avg_chemreg_adam_best.tex}
\input{chemRegB_14042023/final_val_loss_mae_avg_chemreg_adam_worst.tex}
\input{chemRegB_14042023/final_val_loss_mae_avg_chemreg_adam_ratios.tex}
\subsection{test loss}
\input{chemRegB_14042023/test_loss_avg_chemreg_adam_best.tex}
\input{chemRegB_14042023/test_loss_avg_chemreg_adam_worst.tex}
\input{chemRegB_14042023/test_loss_avg_chemreg_adam_ratios.tex}
\end{landscape}
\end{document}